\documentclass[10pt]{article}
\usepackage{geometry}
\usepackage[portuguese]{babel}
\geometry{a4paper} 
\usepackage{graphicx}
\usepackage{amsmath}
\usepackage{amssymb}
\usepackage{fancyhdr}
\fancyhf{}
\pagestyle{fancy}
\fancyfoot[LE,RO]{\thepage}
\setlength\headheight{26pt}
\rhead{\includegraphics[width=4cm]{template-FEI/FEI_logo.png}}
\usepackage{xcolor}

\begin{document}
\noindent \textbf{Centro Universitário FEI}\\
\noindent \textbf{CC6112 - Computação Gráfica}\\
\noindent \textbf{Aluno: } João Pedro Rosa Cezarino  \\ 
\noindent \textbf{R.A: } 22.120.021-5\\
\today
\\
\begin{center}
    \noindent \textbf{Resolução da Atividade 01 - Transformações Geométricas}
\end{center}
\vspace{0.5cm}
\noindent\textbf{Questão 01:}\\
Seja um ponto \emph{P=(2, -1, 3)} no espaço euclidiano 3D. Para deslocá-lo 20 unidades na direção X, 5 unidades em Y e 4 unidades em Z, qual a matriz de trasnformação que devo multiplicar P? Apresente a Matriz e o resultado final separadamente.\\
\\
\noindent\textbf{Solução:}
\[
\text{Matriz que deve multiplicar P:}
	\begin{bmatrix}
        1 & 0 & 0 & 20\\
        0 & 1 & 0 & 5\\
        0 & 0 & 1 & 4\\
        0 & 0 & 0 & 1
    \end{bmatrix}
\]
\\
\[  
\text{Resultado final:}
	\begin{bmatrix}
        22\\
        4\\
        7\\
        1
    \end{bmatrix}
\]


\vspace{1 cm}
\noindent\textbf{Questão 02:}\\
Considerando o mesmo ponto P da questão 1, qual matriz devo aplicar a P para que ele se desloque para o ponto \emph{P=(1,6,4)}?\\
\\
\noindent\textbf{Solução:}
\[
\text{Matriz que deve ser aplicada à P:}
	\begin{bmatrix}
        1 & 0 & 0 & -1\\
        0 & 1 & 0 & 7\\
        0 & 0 & 1 & 1\\
        0 & 0 & 0 & 1
    \end{bmatrix}
\]

\vspace{1 cm}
\noindent\textbf{Questão 03:}\\
Supondo ainda o mesmo ponto \emph{P} da 1º questão, qual a única matriz que devo aplicar a P, para que ele sofra ao mesmo tempo uma rotação de 25º em torno do centro do sistema de coordenadas, seguido de uma translação.\\
\\
\noindent\textbf{Solução:}
\\
$$ T \cdot R = P \rightarrow T \cdot R = M \therefore M \cdot P $$
\\
\[
\text{Matriz que deve ser aplicada à P:}
	\begin{bmatrix}
        0.906 & -0.423 & 0 & 20\\
        0.423 & 0.906 & 0 & 5\\
        0 & 0 & 1 & 4\\
        0 & 0 & 0 & 1
    \end{bmatrix}
\]

\vspace{1cm}
\noindent\textbf{Questão 04:}\\
Seja uma nuvem de pontos no espaço Euclidiano 3D,\emph{P = (2, -1, 1),( 1, 1, 1), (3, 2, 1),( 2, 4, 2),
(1, 0, 2),( 2, 1, 3),( 3, 1, 1), (4, 2, -3),(3, 5, 5), (2, 3, 5)}. Para rotacionar essa nuvem de 75º no eixo X, 45º no eixo Y e 60º no eixo Z, sempre em torno do seu centro de massa, qual a matriz que devo aplicar sobre a nuvem P?\\
\\
\noindent\textbf{Solução:}
\\
$$ M =  T \cdot R_x \cdot R_y \cdot R_z \cdot T' \cdot N$$
\\
\[
\text{M = }
	\begin{bmatrix}
	    0.354 & -0.566 & 0.745 & 0.124\\
        0.612 & -0.462 & -0.641 & 1.80\\
        0.707 &  0.683 & 0.183 & -0.198\\
        0 & 0 & 0 & 1
    \end{bmatrix}
\]

\vspace{1cm}
\noindent\textbf{Questão 05:}\\
Seja a Figura 1 a seguir no plano \emph{x-y}, cujas coordenadas de apenas quatro pontos \emph{(P 1, P 2, P 3 e P 4)}
sao conhecidas. Sabemos que o centro de gravidade da figura coincide com o centro do sistema
de coordenadas \emph{(0, 0)}. Considere \emph{P1 = (11, 8), P2 = (15, -10), P3 = ( -8, -8), P4 = (-14, 9)}.
Responda:
\begin{itemize}
\item \textbf{(a)} Qual a matriz que provoca um contração por igual de 10\% no desenho?
\item \textbf{(b)} Qual a matriz que provoca uma contração de 10\% na dimensão x e 15\% na dimensão y?
\item \textbf{(c)} Quais os valores finais de \emph{P1, P2, P3 e P4}, após a aplicação do item (b)?
\item \textbf{(d)} Se o valor final de \emph{P1}, após uma expansão foi \emph{P1'= (30, 12)}, qual o valor de \emph{P3}?
\item \textbf{(e)} Se o desenho foi primeiro rotacionado de 65º no eixo x, seguido de uma expansão por igual de 50\% no mesmo eixo, qual é a matriz de transformação que substitui as duas operações?
\item \textbf{(f)} Qual a matriz que causará uma reflexão do desenho em torno do ponto de origem? Quais os
valores finais dessa reflexão para cada ponto \emph{P1, P2, P3 e P4}?
\end{itemize}
\\
\\
\noindent\textbf{Solução:}
\\
\begin{itemize}
\item\textbf{(a)} 
\text{Matriz que provoca uma contração de 10\% no desenho:}
\[  T = 
	\begin{bmatrix}
	0.9 & 0 & 0 & 0\\
    0 & 0.9 & 0 & 0\\
    0 & 0 & 1 & 0\\
    0 & 0 & 0 & 1
    \end{bmatrix}
\]
\\
\item\textbf{(b)} 
\text{Matriz que provoca uma contração de 10\% na dimensão x e 15\% na dimensão y:}
\[  M = 
	\begin{bmatrix}
	0.9 & 0 & 0 & 0\\
    0 & 0.85 & 0 & 0\\
    0 & 0 & 1 & 0\\
    0 & 0 & 0 & 1
    \end{bmatrix}
\]
\\
\item\textbf{(c)} 
\text{Após a aplicação do item \emph{b}, os valores finais de \emph{P1, P2, P3 e P4} serão:}
\[\text\emph{P_1=(9.9, 6.8)}\]
\[\text\emph{P_2=(13.5, -8.5)}\]
\[\text\emph{P_3=(-7.2, 6.8)}\]
\[\text\emph{P_4=(-12.6, 7.65)}
\]
\\
\item\textbf{(d)} 
\text{O valor final de \emph{P3} é: }
\[\text\emph{P_3=(-21.6, -12)}\]
\\
\item\textbf{(e)} 
\text{A Matriz que substitui as duas operações é:}
\[  M = 
	\begin{bmatrix}
	1.5 & 0 & 0 & 0\\
    0 & 0.423 & -0.906 & 0\\
    0 & 0.906 & 0.423 & 0\\
    0 & 0 & 0 & 1
    \end{bmatrix}
\]
\\
\item\textbf{(f)} 
\text{A Matriz que causará uma reflexão no desenho em torno do ponto de origem é:}
\[  P = 
	\begin{bmatrix}
	-1 & 0 & 0 & 0\\
    0 & -1 & 0 & 0\\
    0 & 0 & -1 & 0\\
    0 & 0 & 0 & 1
    \end{bmatrix}
\]
\\
\text{Os valores finais dessa reflexão em cada ponto são:}
\[\text\emph{P_1=(-11, -8) }\]
\[\text\emph{P_2=(-15, 10) }\]
\[\text\emph{P_3=(8, 8) }\]
\[\text\emph{P_4=(14, -9) }\]
\end{itemize}
\vspace{1cm}
\noindent\textbf{Questão 06:}\\
Mostre que a composição (concatenação) de duas rotações $\Theta$1 e $\Theta$2 é aditiva, ou seja: \emph{R($\Theta$1)R($\Theta$2) = R($\Theta$1 + $\Theta$2)}.\\
\\
\noindent\textbf{Solução:}
\\
\[
	\begin{bmatrix}
        \cos{$\theta$_1} & -\sin{$\theta$_1} & 0 & 0\\
        \sin{$\theta$_1} & \cos{$\theta$_1} & 0 & 0\\
        0 & 0 & 1 & 0\\
        0 & 0 & 0 & 1
    \end{bmatrix}
    \cdot  
    \begin{bmatrix}
        \cos{$\theta$_2} & -\sin{$\theta$_2} & 0 & 0\\
        \sin{$\theta$_2} & \cos{$\theta$_2} & 0 & 0\\
        0 & 0 & 1 & 0\\
        0 & 0 & 0 & 1
    \end{bmatrix}
    =
    \begin{bmatrix}
        \cos{$\theta$_1 + $\theta$_2} & -\sin{$\theta$_1 + $\theta$_2} & 0 & 0\\
        \sin{$\theta$_1 + $\theta$_2} & \cos{$\theta$_1 + $\theta$_2} & 0 & 0\\
        0 & 0 & 1 & 0\\
        0 & 0 & 0 & 1
    \end{bmatrix}
\]

\vspace{1cm}
\noindent\textbf{Questão 07:}\\
Prove que a multiplicação de duas matrizes de translação aplicadas sucessivamente é comutativa.\\
\\
\noindent\textbf{Solução:}
\[
    M_1 =
	\begin{bmatrix}
        1 & 0 & 0 & \emph{a}\\
        0 & 1 & 0 & \emph{b}\\
        0 & 0 & 1 & \emph{c}\\
        0 & 0 & 0 & 1
    \end{bmatrix}
    \cdot  
    \begin{bmatrix}
        1 & 0 & 0 & \emph{x}\\
        0 & 1 & 0 & \emph{y}\\
        0 & 0 & 1 & \emph{z}\\
        0 & 0 & 0 & 1
    \end{bmatrix}
    =
    \begin{bmatrix}
        1 & 0 & 0 & \emph{a+x}\\
        0 & 1 & 0 & \emph{b+y}\\
        0 & 0 & 1 & \emph{c+z}\\
        0 & 0 & 0 & 1
    \end{bmatrix}
\]
\[
    M_2 =
	\begin{bmatrix}
        1 & 0 & 0 & \emph{x}\\
        0 & 1 & 0 & \emph{y}\\
        0 & 0 & 1 & \emph{z}\\
        0 & 0 & 0 & 1
    \end{bmatrix}
    \cdot  
    \begin{bmatrix}
        1 & 0 & 0 & \emph{a}\\
        0 & 1 & 0 & \emph{b}\\
        0 & 0 & 1 & \emph{c}\\
        0 & 0 & 0 & 1
    \end{bmatrix}
    =
    \begin{bmatrix}
        1 & 0 & 0 & \emph{x+a}\\
        0 & 1 & 0 & \emph{y+b}\\
        0 & 0 & 1 & \emph{z+c}\\
        0 & 0 & 0 & 1
    \end{bmatrix}
\]

\vspace{1cm}
\noindent\textbf{Questão 08:}\\
Mostre que uma rotação R no eixo Z e um escalonamento S no plano \emph{(x, y)} são comutativos
apenas se o escalonamento for uniforme.\\
\\
\noindent\textbf{Solução:}
\\
\[
    M_1 =
    \begin{bmatrix}
        \cos{$\theta$_1} & -\sin{$\theta$_1} & 0 & 0\\
        \sin{$\theta$_1} & \cos{$\theta$_1} & 0 & 0\\
        0 & 0 & 1 & 0\\
        0 & 0 & 0 & 1
    \end{bmatrix}
    \cdot  
    \begin{bmatrix}
        a & 0 & 0 & 0\\
        0 & b & 0 & 0\\
        0 & 0 & 1 & 0\\
        0 & 0 & 0 & 1
    \end{bmatrix}
    =
    \begin{bmatrix}
        a\cdot\cos{$\theta$_1} & b\cdot-\sin{$\theta$_1} & 0 & 0\\
        a\cdot\sin{$\theta$_1} & b\cdot\cos{$\theta$_1} & 0 & 0\\
        0 & 0 & 1 & 0\\
        0 & 0 & 0 & 1
    \end{bmatrix}
\]
\\
\[
    M_2 =
    \begin{bmatrix}
        a & 0 & 0 & 0\\
        0 & b & 0 & 0\\
        0 & 0 & 1 & 0\\
        0 & 0 & 0 & 1
    \end{bmatrix}
    \cdot
    \begin{bmatrix}
        \cos{$\theta$_1} & -\sin{$\theta$_1} & 0 & 0\\
        \sin{$\theta$_1} & \cos{$\theta$_1} & 0 & 0\\
        0 & 0 & 1 & 0\\
        0 & 0 & 0 & 1
    \end{bmatrix}
    =
    \begin{bmatrix}
        a\cdot\cos{$\theta$_1} & a\cdot-\sin{$\theta$_1} & 0 & 0\\
        b\cdot\sin{$\theta$_1} & b\cdot\cos{$\theta$_1} & 0 & 0\\
        0 & 0 & 1 & 0\\
        0 & 0 & 0 & 1
    \end{bmatrix}
\]
\\
\[
    \text{ Logo, $M_1=M_2$ apenas se o escalonamento for uniforme.}
\]

\vspace{1cm}
\noindent\textbf{Questão 09:}\\
Qual a razão de se adotar coordenadas homogêneas em computação gráfica ?\\
\\
\noindent\textbf{Solução:}
\\
O uso de Coordenadas Homogêneas em computação Gráfica tem o objetivo de agilizar 
o cálculo  de transformações, permitindo assim, representar tudo da mesma forma.

\vspace{1cm}
\noindent\textbf{Questão 10:}\\
Seja a transformação abaixo na Figura 1, cujos pontos iniciais e finais são indicados por setas,
dada por uma matriz de escalonamento. Mostre essa matriz.\\
\\
\noindent\textbf{Solução:}
\\

\text{Matriz de escalonamento do eixo x:}
\[
  M_1 = 
  \begin{bmatrix}
        3 & 0 & 0 & 0\\
        0 & 1 & 0 & 0\\
        0 & 0 & 1 & 0\\
        0 & 0 & 0 & 1
    \end{bmatrix}
\]

\vspace{1cm}
\noindent\textbf{Questão 11:}\\
Se considerarmos o ponto \emph{P = (A + B)/2} como o centro de rotação, como faríamos para
representar a rotação de θ graus no sentido anti-horário dos objetos A e B em torno de P e ao
mesmo tempo em torno de seus próprios centros de massa, também de θ graus ?\\
\\
\noindent\textbf{Solução:}
\\
\[ 
    TP' = 
	\begin{bmatrix}
        1 & 0 & 0 & -$\Delta$$x_p$\\
        0 & 1 & 0 & -$\Delta$$y_p$\\
        0 & 0 & 1 & -$\Delta$$z_p$\\
        0 & 0 & 0 & 1
    \end{bmatrix}
    ""
    R_\theta = 
    \begin{bmatrix}
        1 & 0 & 0 & 0\\
        0 & $\cos{\theta}$ & $-\sin{\theta}$ & 0\\
        0 & $\sin{\theta}$ & $\cos{\theta}$ & 0\\
        0 & 0 & 0 & 1
    \end{bmatrix}
\]
\\
\[ 
    TP = 
	\begin{bmatrix}
        1 & 0 & 0 & $\Delta$$x_p$\\
        0 & 1 & 0 & $\Delta$$y_p$\\
        0 & 0 & 1 & $\Delta$$z_p$\\
        0 & 0 & 0 & 1
    \end{bmatrix}
    ""
    TB' = 
    \begin{bmatrix}
        1 & 0 & 0 & -$\Delta$$x_b$\\
        0 & 1 & 0 & -$\Delta$$y_b$\\
        0 & 0 & 1 & -$\Delta$$z_b$\\
        0 & 0 & 0 & 1
    \end{bmatrix}
\]
\\
\[ 
    TB = 
	\begin{bmatrix}
        1 & 0 & 0 & $\Delta$$x_b$\\
        0 & 1 & 0 & $\Delta$$y_b$\\
        0 & 0 & 1 & $\Delta$$z_b$\\
        0 & 0 & 0 & 1
    \end{bmatrix}
    ""
    TA' = 
    \begin{bmatrix}
        1 & 0 & 0 & -$\Delta$$x_a$\\
        0 & 1 & 0 & -$\Delta$$y_a$\\
        0 & 0 & 1 & -$\Delta$$z_a$\\
        0 & 0 & 0 & 1
    \end{bmatrix}
\]
\\
\[ 
    TA = 
	\begin{bmatrix}
        1 & 0 & 0 & $\Delta$$x_a$\\
        0 & 1 & 0 & $\Delta$$y_a$\\
        0 & 0 & 1 & $\Delta$$z_a$\\
        0 & 0 & 0 & 1
    \end{bmatrix}
\]
\\
$$M_1 = TB \cdot R\theta \cdot TB' \cdot TP \cdot R\theta \cdot TP'$$
$$M_2 = TA \cdot R\theta \cdot TA' \cdot TP \cdot R\theta \cdot TP'$$

\vspace{1cm}
\noindent\textbf{Questão 12:}\\
Existe um objeto 2D no plano \emph{x-y} e cujo centro de massa está localizado no ponto \emph{(0, 0)}, sendo
observado por uma câmera localizada no ponto \emph{P1=(x1, y1)}. Qual a matriz de transformação da
câmera que fará com que ela mova-se no plano \emph{x-y} para o ponto \emph{P2=(x2, y2)} que está diametralmente oposto ao ponto P1?\\
\\
\noindent\textbf{Solução:}
\\

\text{Matriz de transformação:}
\[
  M_1 = 
  \begin{bmatrix}
        -1 & 0 & 0 & 0\\
        0 & -1 & 0 & 0\\
        0 & 0 & 1 & 0\\
        0 & 0 & 0 & 1
    \end{bmatrix}
\]
\end{document}

