\begin{filecontents*}{\jobname.xmpdata}
    \Title     {Atividade 09 – Animação de Fluídos}
    \Author    {João Pedro Rosa Cezarino}
    \Keywords  {CC6112\sep OpenGL\sep Cameras\sep FEI}
    \Language  {pt-BR}
    \Subject   {Resolução da Atividade 09 – Animação de Fluídos - CC6112}
\end{filecontents*}

\documentclass[a4paper, 12pt]{article}
\usepackage[utf8]{inputenc}
\usepackage[bottom=3cm,top=2.5cm,left=2cm,right=2cm]{geometry}
\usepackage[brazil]{babel}
\usepackage{graphicx} 
\usepackage{amsmath}
\usepackage{amssymb}
\usepackage{fancyhdr}
\usepackage{xcolor}
\fancyhf{}
\pagestyle{fancy}
\fancyfoot[LE,RO]{\thepage}
\setlength\headheight{26pt}
\rhead{\includegraphics[width=4cm]{template-FEI/FEI_logo.png}}

\begin{document}
\noindent \textbf{Centro Universitário FEI}\\
\noindent \textbf{CC6112 - Computação Gráfica}\\
\noindent \textbf{Aluno: } João Pedro Rosa Cezarino  \\ 
\noindent \textbf{R.A: } 22.120.021-5\\
\today
\\
\begin{center}
    \noindent \textbf{Resolução da Atividade 09 – Animação de Fluídos}
\end{center}

\vspace{0.5cm}
\noindent\textbf{Questão 01:}\\
Considerando uma subdivisão espacial para a simulação de um Sistema de Partículas
baseada em SPH, explique porque consideramos um kernel gaussiano para a sua
interpolação?\\
\\
\noindent\textbf{Solução:}

Uma vez que desejamos que as partículas mais próximas da posição central \emph{r} tenham mais influência ou contribuam no cálculo da propriedade dentro do kernel, a função gaussiana é a mais apropriada para essa tarefa. Sabendo que a função gaussiana tem a média e o desvio padrão como variáveis, e considerando a posição \emph{r} como a média do SPH, apenas o desvio padrão influenciará nas particulas vizinhas. 

Portanto, quanto maior o desvio padrão, mais partículas vizinhas terão a posição \emph{r}. Quanto menor o desvio padrão, menos partículas vizinhas existirão. Vale lembrar que o decaimento da curva  é importante, pois quanto mais distantes do topo as partículas estão, menos interferência elas possuem com a partícula central.
\vspace{1cm}

\noindent\textbf{Questão 02:}\\
Quais os passos, na sequência correta da renderização, do método SPH para
implementação de animação de fluídos?\\
\\
\noindent\textbf{Solução:}

Para a implementação de animação de fluídos, devemos calcular algumas propriedades da partícula como:  Força de Pressão (F\textsubscript{p}), Força de Viscosidade (F\textsubscript{v}) e a  Força de gravidade (F\textsubscript{g}). Assim, em um loop, a cada iteração \emph{T} calcularemos essas forças e faremos a sua composição, assim como demonstrado abaixo.
\[\mathbf{Q^t_i = F_p + F_g + F_v}\]

Após isto, calculamos a Aceleração e a Velocidade e em seguida a nova posição da partícula. Por
fim, as partículas vizinhas devem ser atualizadas.
\vspace{0.5cm}

\noindent\textbf{Questão 03:}\\
Cite pelo menos duas razões para se fazer animação baseada em sistemas de
partículas, dispensando-se a facilidade de desenhar quadro a quadro manualmente
o escoamento do fluído.\\
\\
\noindent\textbf{Solução:}

Num sistema de partículas menos recursos computacionais são exigidos, quando comparado a uma simulação de molécula a molécula. Além disso, no sistema de particulas também é mais fácil de se apresentar a interação entre as partículas e as simulações tornam-se mais realistas e eficientes.
\vspace{1cm}

\noindent\textbf{Questão 04:}\\
Uma saída estratégica para diminuir a complexidade computacional de um algoritmo de
simulação de fluídos que utiliza o SPH é manter uma lista de vizinhança $Li$ para cada
partícula $i$ dentro da subdivisão espacial 3D. Justifique a complexidade computacional do
algoritmo de simulação de fluídos que utiliza essa estratégia.\\
\\
\noindent\textbf{Solução:}

Como cada célula possui uma lista com as suas partículas vizinhas, a busca é realizada apenas na célula onde a partícula é encontrada e em suas células vizinhas. Essa estratégia, reduz o custo computacional de $\mathbf{O(n^2)}$, para $\mathbf{O(n \cdot m)}$, onde \textbf{n}  ́e o número total de partículas e \textbf{m} é o número médio de partículas nas células vizinhas.
\vspace{1cm}

\noindent\textbf{Questão 05:}\\
Para o algoritmo SPH, os seguintes cálculos devem ser executados (não
necessariamente nessa ordem): \textit{Velocidade, Aceleração, Pressão, Atualização da
nova Vizinhança , Posição da Partícula, Viscosidade, Velocidade}. Proponha uma
ordem que faça sentido e explique o porquê.\\
\\
\noindent\textbf{Solução:}

\noindent\textbf{Ordem Proposta:}

\[Pressão \rightarrow Viscosidade \rightarrow Aceleração \rightarrow Velocidade \rightarrow \text{Posição da Partícula}\]
\[\rightarrow \text{Atualização da nova Vizinhança.}\]

Esta ordem deve ser seguida, pois é necessário primeiramente calcular as propriedades físicas da partícula (Pressão e Viscosidade) e logo após, é necessário descobrir a que ponto a partícula se direciona (Aceleração, Velocidade e Posição da Partícula). Por fim, atualizam-se as partículas vizinhas, já que estas sofreram influência da partícula em questão.Vale lembrar que é comum considerarmos a mesma massa para todas as partículas.
\vspace{1cm}

\noindent\textbf{Questão 06:}\\
Explique o porquê uma simulação de fluídos não representa estritamente a realidade física.\\
\\
\noindent\textbf{Solução:}

Uma simulação de Fluídos não representa a realidade física porque atualmente é impossível  simular molécula por molécula, devido à isto, simulamos um sistema de partículas inteiro.

\vspace{1cm}

\noindent\textbf{Questão 07:}\\
Pense em algum problema que possa fazer uso as simulação de fluídos, mas não necessariamente é fluído. \\
\\
\noindent\textbf{Solução:}

Podemos fazer a simulação de fluídos em uma piscina de bolinhas, por exemplo: Neste caso, se as bolinhas estivessem se mexendo,  seria possivel encontrar o fluxo delas. Outro exemplo, seriam em Deslizamento de encostas e Avalanches.
\end{document}
