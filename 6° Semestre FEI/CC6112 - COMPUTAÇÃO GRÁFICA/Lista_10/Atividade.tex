\begin{filecontents*}{\jobname.xmpdata}
    \Title     {Atividade 10 - Animação de Tecidos}
    \Author    {João Pedro Rosa Cezarino}
    \Keywords  {CC6112\sep OpenGL\sep Cameras\sep FEI}
    \Language  {pt-BR}
    \Subject   {Resolução da Atividade 10 - Animação de Tecidos}
\end{filecontents*}

\documentclass[a4paper, 12pt]{article}
\usepackage[utf8]{inputenc}
\usepackage[bottom=3cm,top=2.5cm,left=2cm,right=2cm]{geometry}
\usepackage[brazil]{babel}
\usepackage{graphicx} 
\usepackage{amsmath}
\usepackage{amssymb}
\usepackage{fancyhdr}
\usepackage{xcolor}
\fancyhf{}
\pagestyle{fancy}
\fancyfoot[LE,RO]{\thepage}
\setlength\headheight{26pt}
\rhead{\includegraphics[width=4cm]{template-FEI/FEI_logo.png}}

\begin{document}
\noindent \textbf{Centro Universitário FEI}\\
\noindent \textbf{CC6112 - Computação Gráfica}\\
\noindent \textbf{Aluno: } João Pedro Rosa Cezarino  \\ 
\noindent \textbf{R.A: } 22.120.021-5\\
\today
\\
\begin{center}
    \noindent \textbf{Resolução da Atividade 10 - Animação de Tecidos}
\end{center}

\vspace{0.5cm}
\noindent\textbf{Questão 01:}

Considere o modelo de animação de tecido dados em sala:

\[m_ia_i + \sum\limits_{j=1}^{n} k \cdot d_i_j + \sum\limits_{j=1}^{n} F_at \cdot a_j - Fg = 0\]

Descreva cada uma das variáveis físicas do modelo.\\
\\
\noindent\textbf{Solução:}
\begin{enumerate}
    \item $\mathbf{m_i} \rightarrow$  Massa das partículas;
    \item $\mathbf{a_i} \rightarrow$  Aceleração resultante sobre a partícula;
    \item $\mathbf{\sum\limits_{j=1}^{n} k \cdot d_i_j} \rightarrow$ Força elástica de Hook, onde \textbf{K} é a constante de Elasticidade;
    \item $\mathbf{\sum\limits_{j=1}^{n} F_at \cdot a_j} \rightarrow$ Força de Atrito. Dependente da aceleração;
    \item $\mathbf{Fg} \rightarrow$ Força Gravitacional.
\end{enumerate}
\vspace{0.5cm}

\noindent\textbf{Questão 02:}

A equação física para o modelo massa-mola é mostrada a seguir:

\[m_ia_i + \sum\limits_{j=1}^{n} k \cdot d_i_j + \sum\limits_{j=1}^{n} F_at \cdot a_j - Fg = 0\]

Onde o primeiro termo corresponde à força de aceleração de cada partícula, o segundo termo à força elástica, o terceiro termo à força de atrito com o ar e o quarto termo à força gravitacional. O algoritmo para a simulação desse modelo, considerando \emph{n} partículas com massa constante, contém vários passos. No entanto, alguns são imprescindíveis e também devem estar na ordem certa. A seguir são mostrados 13 comandos em pseudocódigo que podem ser utilizados para a implementação dessa simulação. Três desses comandos são desnecessários; os demais são mostrados em uma ordem que não está correta. Enumere na coluna da esquerda os 10 comandos, na ordem correta para que o algoritmo funcione.\\
\newpage
\noindent\textbf{Solução:}
\begin{table}[h!]
    \centering
        \begin{tabular}{|| c | c ||}
        \hline
         7 & Calcular a posição da partícula \\ \hline
         3 & Calcular a força de atrito \\ \hline
         2 & Para Cada Partícula, faça \\ \hline
         - & Calcular o Número de Partículas \\ \hline
         9 & Renderizar a cena \\ \hline
         8 & Fim Para \\ \hline
         1 & Para cada instante de tempo \emph{t}, faça \\ \hline
         6 & Calcular a Velocidade \\ \hline
         4 & Calcular a Força de Elasticidade \\ \hline
         - & Calcular a Massa de Cada Partícula \\ \hline
         10 & Fim Para \\ \hline
         5 & Calcular a Aceleração \\ \hline
         - & Calcular o Choque entre as Partículas \\ \hline
        \end{tabular}
\end{table}
\vspace{0.5cm}

\noindent\textbf{Questão 03:}

No caso de animação de tecidos, deve-se calcular diversas propriedades físicas. Considerando um algoritmo sequencial para os cálculos das propriedades \textbf{i) forças de elasticidade}; \textbf{ii) posição de cada partícula}; \textbf{iii) velocidade} e \textbf{iv) aceleração do tecido}, qual é a melhor ordem para o cálculo de \textbf{i} a \textbf{iv} de cada propriedade para realizar a animação adequada de cada partícula ?\\
\\
\noindent\textbf{Solução:}
\begin{enumerate}
    \item Cálculo da Força Elástica;
    \item Cálculo da Acerelação do Tecido;
    \item Cálculo da Velocidade;
    \item Cálculo da Posição de cada partícula.
\end{enumerate}
\vspace{0.5cm}

\noindent\textbf{Questão 04:}
\begin{center}
    \textbf{Implementação Prática em anexo junto à este documento.}
\end{center}
\end{document}
